\chapter{Opérations sur les tables de données}

\section{Travail sur les lignes et colonnes}

Dans une grande variété de situations, il peut être avantageux de répéter une opération sur toutes les lignes, ou toutes les colonnes.
R propose une fonction pour automatiser ce traitement, \emph{via} la fonction \texttt{apply}. 

\begin{knitrout}
\definecolor{shadecolor}{rgb}{0.969, 0.969, 0.969}\color{fgcolor}\begin{kframe}
\begin{flushleft}
\ttfamily\noindent
\hlsymbol{dat}{\ }\hlassignement{=}{\ }\hlfunctioncall{matrix}\hlkeyword{(}\hlfunctioncall{rnorm}\hlkeyword{(}\hlnumber{100}\hlkeyword{)}\hlkeyword{,}{\ }\hlargument{nrow}{\ }\hlargument{=}{\ }\hlnumber{10}\hlkeyword{)}\hspace*{\fill}\\
\hlstd{}\hlfunctioncall{apply}\hlkeyword{(}\hlsymbol{dat}\hlkeyword{,}{\ }\hlnumber{1}\hlkeyword{,}{\ }\hlsymbol{mean}\hlkeyword{)}\mbox{}
\normalfont
\end{flushleft}
\begin{verbatim}
##  [1] -0.274393  0.109191 -0.008739  0.111188  0.243100  0.037100  0.061842
##  [8] -0.207347 -0.027129  0.550166
\end{verbatim}
\begin{flushleft}
\ttfamily\noindent
\hlfunctioncall{apply}\hlkeyword{(}\hlsymbol{dat}\hlkeyword{,}{\ }\hlnumber{2}\hlkeyword{,}{\ }\hlsymbol{var}\hlkeyword{)}\mbox{}
\normalfont
\end{flushleft}
\begin{verbatim}
##  [1] 0.9487 1.3005 0.5893 0.7521 1.3198 0.5092 0.5799 2.2701 1.2619 0.3809
\end{verbatim}
\end{kframe}
\end{knitrout}


\section{Traitement des données}

\section{Division et traitement par niveau}
