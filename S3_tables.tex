\chapter{Opérations sur les tables de données\label{c:tables}}

\section{Travail sur les lignes et colonnes}

Dans une grande variété de situations, il peut être avantageux de répéter une opération sur toutes les lignes, ou toutes les colonnes.
R propose une fonction pour automatiser ce traitement, \emph{via} la fonction \texttt{apply}. 

\begin{knitrout}
\definecolor{shadecolor}{rgb}{1, 1, 1}\color{fgcolor}\begin{kframe}
\begin{flushleft}
\ttfamily\noindent
\hlsymbol{dat}{\ }\hlassignement{\usebox{\hlnormalsizeboxlessthan}-}{\ }\hlfunctioncall{matrix}\hlkeyword{(}\hlfunctioncall{rnorm}\hlkeyword{(}\hlnumber{100}\hlkeyword{)}\hlkeyword{,}{\ }\hlargument{nrow}{\ }\hlargument{=}{\ }\hlnumber{10}\hlkeyword{)}\hspace*{\fill}\\
\hlstd{}\hlfunctioncall{apply}\hlkeyword{(}\hlsymbol{dat}\hlkeyword{,}{\ }\hlnumber{1}\hlkeyword{,}{\ }\hlsymbol{mean}\hlkeyword{)}\mbox{}
\normalfont
\end{flushleft}
\begin{verbatim}
##  [1]  0.01547 -0.37188  0.29644 -0.08430  0.04086  0.09668  0.38595
##  [8]  0.48176 -0.07378  0.35627
\end{verbatim}
\begin{flushleft}
\ttfamily\noindent
\hlfunctioncall{apply}\hlkeyword{(}\hlsymbol{dat}\hlkeyword{,}{\ }\hlnumber{2}\hlkeyword{,}{\ }\hlsymbol{var}\hlkeyword{)}\mbox{}
\normalfont
\end{flushleft}
\begin{verbatim}
##  [1] 0.8647 1.8382 0.8380 1.8345 0.9208 1.2329 1.0624 1.0277 0.3362 0.9321
\end{verbatim}
\end{kframe}
\end{knitrout}


\section{Division et traitement par niveau}

En utilisant différentes fonctions, on peut traiter facilement un jeu de données par «niveaux» d'un facteur (p.ex. traitement expérimental).
En rechargeant les données \emph{Lamellodiscus}, on peut par exemple chercher à connaître la moyenne et la variance de la taille de chaque pièce sclérifiée.

\begin{knitrout}
\definecolor{shadecolor}{rgb}{1, 1, 1}\color{fgcolor}\begin{kframe}
\begin{flushleft}
\ttfamily\noindent
\hlsymbol{morpho}{\ }\hlassignement{\usebox{\hlnormalsizeboxlessthan}-}{\ }\hlfunctioncall{read.table}\hlkeyword{(}\hlstring{"{}data/lamellodiscus.txt"{}}\hlkeyword{,}{\ }\hlargument{h}{\ }\hlargument{=}{\ }\hlnumber{TRUE}\hlkeyword{,}{\ }\hlargument{sep}{\ }\hlargument{=}{\ }\hlstring{"{}\usebox{\hlnormalsizeboxbackslash}t"{}}\hlkeyword{)}\mbox{}
\normalfont
\end{flushleft}
\end{kframe}
\end{knitrout}


\noindent L'étape suivante est de diviser les données, en utilisant la fonction \texttt{split}.
Cette fonction prend une \texttt{data.frame}, la divise selon les valeurs de la colonne (ou combinaison de colonnes) choisie, et renvoie les sous-tableaux sous forme de liste.

\begin{knitrout}
\definecolor{shadecolor}{rgb}{1, 1, 1}\color{fgcolor}\begin{kframe}
\begin{flushleft}
\ttfamily\noindent
\hlsymbol{morpho\usebox{\hlnormalsizeboxunderscore}split}{\ }\hlassignement{\usebox{\hlnormalsizeboxlessthan}-}{\ }\hlfunctioncall{split}\hlkeyword{(}\hlsymbol{morpho}\hlkeyword{,}{\ }\hlsymbol{morpho}\hlkeyword{\usebox{\hlnormalsizeboxdollar}}\hlsymbol{sppar}\hlkeyword{)}\hspace*{\fill}\\
\hlstd{}\hlfunctioncall{names}\hlkeyword{(}\hlsymbol{morpho\usebox{\hlnormalsizeboxunderscore}split}\hlkeyword{)}\mbox{}
\normalfont
\end{flushleft}
\begin{verbatim}
##  [1] "conf" "dipl" "eleg" "erge" "falc" "frat" "furc" "igno" "kech" "morm"
## [11] "neif" "ther" "tome"
\end{verbatim}
\begin{flushleft}
\ttfamily\noindent
\hlsymbol{morpho\usebox{\hlnormalsizeboxunderscore}split}\hlkeyword{\usebox{\hlnormalsizeboxdollar}}\hlsymbol{conf}\mbox{}
\normalfont
\end{flushleft}
\begin{verbatim}
##     sphote sppar       para     a     b      c      d      f      g     aa
## 153   Sasa  conf Sasa1conf1 1.300 1.250 0.5500 0.3000 0.3500 0.2000 1.2000
## 188   Disa  conf Disa9conf1 1.204 1.244 0.7498 0.7443 0.6972 0.3702 0.9759
##         bb     cc    lm    li
## 153 1.1000 0.6500 1.300 1.300
## 188 0.9007 0.6713 1.173 1.144
\end{verbatim}
\end{kframe}
\end{knitrout}


\noindent On obtient 13 tableaux de données, un pour chaque espèce de parasites.
On souhaite éliminer ceux qui ont été observés moins de trois fois au total.
Ceci implique de parcourir chaque élément de la liste, et de déterminer sa taille.
R propose une fonction \texttt{lapply}, littéralement \texttt{apply} sur une \texttt{l}iste, pour effectuer cette tâche:

\begin{knitrout}
\definecolor{shadecolor}{rgb}{1, 1, 1}\color{fgcolor}\begin{kframe}
\begin{flushleft}
\ttfamily\noindent
\hlsymbol{n\usebox{\hlnormalsizeboxunderscore}obs}{\ }\hlassignement{\usebox{\hlnormalsizeboxlessthan}-}{\ }\hlfunctioncall{unlist}\hlkeyword{(}\hlfunctioncall{lapply}\hlkeyword{(}\hlsymbol{morpho\usebox{\hlnormalsizeboxunderscore}split}\hlkeyword{,}{\ }\hlsymbol{nrow}\hlkeyword{)}\hlkeyword{)}\hspace*{\fill}\\
\hlstd{}\hlsymbol{n\usebox{\hlnormalsizeboxunderscore}obs}{\ }\hlkeyword{\usebox{\hlnormalsizeboxgreaterthan}=}{\ }\hlnumber{3}\mbox{}
\normalfont
\end{flushleft}
\begin{verbatim}
##  conf  dipl  eleg  erge  falc  frat  furc  igno  kech  morm  neif  ther 
## FALSE FALSE  TRUE  TRUE  TRUE  TRUE  TRUE  TRUE  TRUE FALSE  TRUE FALSE 
##  tome 
##  TRUE 
\end{verbatim}
\end{kframe}
\end{knitrout}


\noindent Notons que \texttt{lapply} retourne une liste.
On peut ensuite utiliser les infomations sur le nombre d'observations, \texttt{n\_obs}, pour choisir quels sous-tableaux garder: 

\begin{knitrout}
\definecolor{shadecolor}{rgb}{1, 1, 1}\color{fgcolor}\begin{kframe}
\begin{flushleft}
\ttfamily\noindent
\hlsymbol{morpho\usebox{\hlnormalsizeboxunderscore}split}{\ }\hlassignement{\usebox{\hlnormalsizeboxlessthan}-}{\ }\hlsymbol{morpho\usebox{\hlnormalsizeboxunderscore}split}\hlkeyword{[}\hlsymbol{n\usebox{\hlnormalsizeboxunderscore}obs}{\ }\hlkeyword{\usebox{\hlnormalsizeboxgreaterthan}=}{\ }\hlnumber{3}\hlkeyword{]}\mbox{}
\normalfont
\end{flushleft}
\end{kframe}
\end{knitrout}


\noindent la encore, on remarquera que pour exclure certains éléments d'une liste, on utilise les crochets simples, comme pour un vecteur, et non les crochets doubles.
On vérifie maintenant qu'il ne reste plus que des espèces avec plus de 3 observations:

\begin{knitrout}
\definecolor{shadecolor}{rgb}{1, 1, 1}\color{fgcolor}\begin{kframe}
\begin{flushleft}
\ttfamily\noindent
\hlfunctioncall{unlist}\hlkeyword{(}\hlfunctioncall{lapply}\hlkeyword{(}\hlsymbol{morpho\usebox{\hlnormalsizeboxunderscore}split}\hlkeyword{,}{\ }\hlsymbol{nrow}\hlkeyword{)}\hlkeyword{)}\mbox{}
\normalfont
\end{flushleft}
\begin{verbatim}
## eleg erge falc frat furc igno kech neif tome 
##   59   19    9    6    7   43   30    6    3 
\end{verbatim}
\end{kframe}
\end{knitrout}


On veut maintenant calculer la moyenne des éléments de chaque sous-tableau, en ne sélectionnant que les colonnes correspondant aux mesures morphométriques.
Ces colonnes sont les 4 et suivantes, soit \texttt{c(4:ncol(x))} dans le langage de R, si on travaille sur un objet \texttt{x}.
Une fois ces colonnes extraites, on peut vérifier qu'on obtient bien une matrice,

\begin{knitrout}
\definecolor{shadecolor}{rgb}{1, 1, 1}\color{fgcolor}\begin{kframe}
\begin{flushleft}
\ttfamily\noindent
\hlsymbol{morpho\usebox{\hlnormalsizeboxunderscore}split}\hlkeyword{\usebox{\hlnormalsizeboxdollar}}\hlsymbol{furc}\hlkeyword{[}\hlkeyword{,}{\ }\hlfunctioncall{c}\hlkeyword{(}\hlnumber{4}\hlkeyword{:}\hlfunctioncall{ncol}\hlkeyword{(}\hlsymbol{morpho\usebox{\hlnormalsizeboxunderscore}split}\hlkeyword{\usebox{\hlnormalsizeboxdollar}}\hlsymbol{furc}\hlkeyword{)}\hlkeyword{)}\hlkeyword{]}\mbox{}
\normalfont
\end{flushleft}
\begin{verbatim}
##         a     b     c      d      f      g    aa    bb     cc    lm    li
## 23  2.030 1.930 1.250 0.5200 0.7300 0.3100 1.880 1.770 1.2000 2.350 2.760
## 140 2.100 1.850 1.500 0.6000 0.7000 0.4000 1.800 1.700 1.2000 1.900 2.700
## 141 2.000 1.900 1.300 0.6000 0.7000 0.3000 1.700 1.600 1.2000 1.800 2.800
## 142 1.900 1.850 1.450 0.6000 0.7500 0.4000 1.850 1.700 1.1000 1.950 2.600
## 175 1.905 1.843 1.305 0.6972 0.7975 0.4873 1.745 1.682 1.1482 2.212 2.621
## 176 1.813 1.691 1.240 0.6617 0.7786 0.5374 1.579 1.477 0.9888 1.693 2.187
## 187 2.031 1.939 1.420 0.7219 0.8022 0.4668 1.832 1.762 1.2685 2.007 2.680
\end{verbatim}
\end{kframe}
\end{knitrout}


\noindent, dont on peut calculer la moyenne sur chaque colonne par la fonction \texttt{apply}.

\begin{knitrout}
\definecolor{shadecolor}{rgb}{1, 1, 1}\color{fgcolor}\begin{kframe}
\begin{flushleft}
\ttfamily\noindent
\hlsymbol{moy}{\ }\hlassignement{\usebox{\hlnormalsizeboxlessthan}-}{\ }\hlkeyword{function}\hlkeyword{(}\hlformalargs{x}\hlkeyword{)}{\ }\hlfunctioncall{apply}\hlkeyword{(}\hlsymbol{x}\hlkeyword{[}\hlkeyword{,}{\ }\hlfunctioncall{c}\hlkeyword{(}\hlnumber{4}\hlkeyword{:}\hlfunctioncall{ncol}\hlkeyword{(}\hlsymbol{x}\hlkeyword{)}\hlkeyword{)}\hlkeyword{]}\hlkeyword{,}{\ }\hlnumber{2}\hlkeyword{,}{\ }\hlsymbol{mean}\hlkeyword{,}{\ }\hlargument{na.rm}{\ }\hlargument{=}{\ }\hlnumber{TRUE}\hlkeyword{)}\mbox{}
\normalfont
\end{flushleft}
\end{kframe}
\end{knitrout}


On peut maintenant appliquer cette fonction à nos données divisées en groupes:

\begin{knitrout}
\definecolor{shadecolor}{rgb}{1, 1, 1}\color{fgcolor}\begin{kframe}
\begin{flushleft}
\ttfamily\noindent
\hlfunctioncall{lapply}\hlkeyword{(}\hlsymbol{morpho\usebox{\hlnormalsizeboxunderscore}split}\hlkeyword{,}{\ }\hlsymbol{moy}\hlkeyword{)}\mbox{}
\normalfont
\end{flushleft}
\begin{verbatim}
## $eleg
##      a      b      c      d      f      g     aa     bb     cc     lm 
## 1.7070 1.6101 1.1589 0.5191 0.5957 0.3475 1.5249 1.4336 0.9794 1.9543 
##     li 
## 2.2411 
## 
## $erge
##      a      b      c      d      f      g     aa     bb     cc     lm 
## 2.0589 1.9603 1.3527 0.7425 0.9609 0.5216 1.7363 1.6743 0.9852 2.6796 
##     li 
## 2.5110 
## 
## $falc
##      a      b      c      d      f      g     aa     bb     cc     lm 
## 1.4151 1.2760 0.6742 0.5014 0.4512 0.2660 1.1984 1.1344 0.7695 1.4033 
##     li 
## 1.3521 
## 
## $frat
##      a      b      c      d      f      g     aa     bb     cc     lm 
## 1.6333 1.4667 1.0333 0.4833 0.6083 0.2917 1.3333 1.1583 0.9333 1.2000 
##     li 
## 1.5000 
## 
## $furc
##      a      b      c      d      f      g     aa     bb     cc     lm 
## 1.9684 1.8575 1.3522 0.6287 0.7512 0.4145 1.7694 1.6700 1.1579 1.9873 
##     li 
## 2.6212 
## 
## $igno
##      a      b      c      d      f      g     aa     bb     cc     lm 
## 1.1418 1.0682 0.6863 0.4650 0.5244 0.2613 1.0015 0.9159 0.5487 1.6268 
##     li 
## 1.4836 
## 
## $kech
##      a      b      c      d      f      g     aa     bb     cc     lm 
## 1.8057 1.7346 1.1192 0.5808 0.8611 0.4013 1.5480 1.4304 0.7800 1.8590 
##     li 
## 1.8269 
## 
## $neif
##      a      b      c      d      f      g     aa     bb     cc     lm 
## 1.0529 0.9435 0.6178 0.4035 0.4394 0.2495 0.8778 0.8019 0.4958 0.7910 
##     li 
## 1.0076 
## 
## $tome
##      a      b      c      d      f      g     aa     bb     cc     lm 
## 2.2833 2.1833 1.2833 0.8000 1.1500 0.3833 1.8167 1.8333 0.9167 2.8167 
##     li 
## 3.0833 
## 
\end{verbatim}
\end{kframe}
\end{knitrout}


On peut aussi convertir facilement cette information en une \texttt{data.frame}, que l'on pivote pour avoir les noms des espèces en lignes:

\begin{knitrout}
\definecolor{shadecolor}{rgb}{1, 1, 1}\color{fgcolor}\begin{kframe}
\begin{flushleft}
\ttfamily\noindent
\hlfunctioncall{t}\hlkeyword{(}\hlfunctioncall{as.data.frame}\hlkeyword{(}\hlfunctioncall{lapply}\hlkeyword{(}\hlsymbol{morpho\usebox{\hlnormalsizeboxunderscore}split}\hlkeyword{,}{\ }\hlsymbol{moy}\hlkeyword{)}\hlkeyword{)}\hlkeyword{)}\mbox{}
\normalfont
\end{flushleft}
\begin{verbatim}
##          a      b      c      d      f      g     aa     bb     cc    lm
## eleg 1.707 1.6101 1.1589 0.5191 0.5957 0.3475 1.5249 1.4336 0.9794 1.954
## erge 2.059 1.9603 1.3527 0.7425 0.9609 0.5216 1.7363 1.6743 0.9852 2.680
## falc 1.415 1.2760 0.6742 0.5014 0.4512 0.2660 1.1984 1.1344 0.7695 1.403
## frat 1.633 1.4667 1.0333 0.4833 0.6083 0.2917 1.3333 1.1583 0.9333 1.200
## furc 1.968 1.8575 1.3522 0.6287 0.7512 0.4145 1.7694 1.6700 1.1579 1.987
## igno 1.142 1.0682 0.6863 0.4650 0.5244 0.2613 1.0015 0.9159 0.5487 1.627
## kech 1.806 1.7346 1.1192 0.5808 0.8611 0.4013 1.5480 1.4304 0.7800 1.859
## neif 1.053 0.9435 0.6178 0.4035 0.4394 0.2495 0.8778 0.8019 0.4958 0.791
## tome 2.283 2.1833 1.2833 0.8000 1.1500 0.3833 1.8167 1.8333 0.9167 2.817
##         li
## eleg 2.241
## erge 2.511
## falc 1.352
## frat 1.500
## furc 2.621
## igno 1.484
## kech 1.827
## neif 1.008
## tome 3.083
\end{verbatim}
\end{kframe}
\end{knitrout}


\section{Traitement des données}

Il existe des moyens de rendre les étapes décrites dans la partie précédente automatique.
Par exemple, la fonction \texttt{aggregate} permet d'aggréger les données en fonction de deux éléments: une combinaison de facteurs, et une fonction.
On peut, en une ligne, connaître la moyenne de chacune des mesures, par hôte et par parasite, avec

\begin{knitrout}
\definecolor{shadecolor}{rgb}{1, 1, 1}\color{fgcolor}\begin{kframe}
\begin{flushleft}
\ttfamily\noindent
\hlfunctioncall{aggregate}\hlkeyword{(}\hlsymbol{morpho}\hlkeyword{,}{\ }\hlargument{by}{\ }\hlargument{=}{\ }\hlfunctioncall{list}\hlkeyword{(}\hlargument{hote}{\ }\hlargument{=}{\ }\hlsymbol{morpho}\hlkeyword{\usebox{\hlnormalsizeboxdollar}}\hlsymbol{sphot}\hlkeyword{,}{\ }\hlargument{parasite}{\ }\hlargument{=}{\ }\hlsymbol{morpho}\hlkeyword{\usebox{\hlnormalsizeboxdollar}}\hlsymbol{sppar}\hlkeyword{)}\hlkeyword{,}\hspace*{\fill}\\
\hlstd{}{\ }{\ }{\ }{\ }\hlsymbol{mean}\hlkeyword{)}\mbox{}
\normalfont
\end{flushleft}
\begin{verbatim}
##    hote parasite sphote sppar para     a     b      c      d      f      g
## 1  Disa     conf     NA    NA   NA 1.204 1.244 0.7498 0.7443 0.6972 0.3702
## 2  Sasa     conf     NA    NA   NA 1.300 1.250 0.5500 0.3000 0.3500 0.2000
## 3  Divu     dipl     NA    NA   NA 1.100 1.025 0.5250 0.4500 0.4750 0.2250
## 4  Dian     eleg     NA    NA   NA 1.650 1.450 0.9750 0.6000 0.6500 0.3000
## 5  Disa     eleg     NA    NA   NA 1.740 1.662 1.1857 0.5340 0.6263     NA
## 6  Divu     eleg     NA    NA   NA 1.760    NA     NA     NA     NA     NA
## 7  Obme     eleg     NA    NA   NA 1.392 1.292 0.8667 0.4250 0.4667 0.3417
## 8  Dipu     erge     NA    NA   NA 2.386 2.314 1.6667 0.8382 1.1478 0.5814
## 9  Disa     erge     NA    NA   NA    NA    NA     NA     NA     NA     NA
## 10 Divu     erge     NA    NA   NA 2.170 2.045 1.1750 0.9250 0.9000 0.6650
## 11 Disa     falc     NA    NA   NA 1.443 0.994 0.5997 0.4850 0.4599 0.2461
## 12 Divu     falc     NA    NA   NA 1.407    NA     NA     NA     NA     NA
## 13 Dian     frat     NA    NA   NA 1.633 1.467 1.0333 0.4833 0.6083 0.2917
## 14 Disa     furc     NA    NA   NA 1.968 1.857 1.3522 0.6287 0.7512 0.4145
## 15 Dipu     igno     NA    NA   NA 1.278 1.274 0.8402 0.6224 0.6856 0.3492
## 16 Disa     igno     NA    NA   NA    NA    NA     NA     NA     NA     NA
## 17 Divu     igno     NA    NA   NA 1.229    NA     NA     NA     NA     NA
## 18 Limo     igno     NA    NA   NA 1.020 0.910 0.5050 0.4350 0.4600 0.2400
## 19 Sasa     igno     NA    NA   NA    NA    NA     NA     NA     NA     NA
## 20 Disa     kech     NA    NA   NA 1.789    NA     NA     NA     NA     NA
## 21 Divu     kech     NA    NA   NA 1.811    NA     NA     NA     NA     NA
## 22 Disa     morm     NA    NA   NA 1.600 1.550 1.2500 0.7000 0.8000 0.6500
## 23 Disa     neif     NA    NA   NA    NA    NA     NA     NA     NA     NA
## 24 Divu     neif     NA    NA   NA 1.815 1.718 1.2890 0.7173 0.7970 0.4476
## 25 Dipu     ther     NA    NA   NA 2.365 2.218 1.5876 0.8393 1.0874 0.5883
## 26 Divu     tome     NA    NA   NA 2.283 2.183 1.2833 0.8000 1.1500 0.3833
##        aa     bb     cc    lm    li
## 1  0.9759 0.9007 0.6713 1.173 1.144
## 2  1.2000 1.1000 0.6500 1.300 1.300
## 3  1.2000 1.0750 0.7000 1.200 1.575
## 4  1.5750 1.4000 0.8250 1.500 2.050
## 5  1.5289 1.4760 0.9987    NA 2.342
## 6  1.6381     NA     NA    NA    NA
## 7  1.1833 1.0250 0.7000 1.267    NA
## 8  1.9896 1.8985 1.1497 2.801 2.639
## 9  1.6516 1.5806 0.9307 2.667 2.406
## 10 1.7800 1.8350 1.0100 2.520 2.940
## 11 0.9029 0.8876 0.5781 1.562 1.297
## 12 1.2829     NA     NA    NA    NA
## 13 1.3333 1.1583 0.9333 1.200 1.500
## 14 1.7694 1.6700 1.1579 1.987 2.621
## 15 1.1818 1.1463 0.7165 1.789 1.774
## 16     NA     NA     NA 1.692 1.497
## 17 1.0137 0.9188 0.6050 1.601    NA
## 18 0.9150 0.7900 0.4700 1.545 1.410
## 19     NA     NA     NA    NA    NA
## 20 1.4686     NA     NA 2.080 1.893
## 21 1.5722     NA     NA    NA    NA
## 22 1.4000 1.3000 0.8000 2.200 2.000
## 23     NA     NA     NA    NA    NA
## 24 1.3888 1.3594 0.9788 1.905 1.838
## 25 1.9996 1.9543 1.1350 2.802 2.812
## 26 1.8167 1.8333 0.9167 2.817 3.083
\end{verbatim}
\end{kframe}
\end{knitrout}


\noindent Cette ligne signifie, en clair, pour chaque niveau de \texttt{sphot} et pour chaque niveau de \texttt{sppar}, calculer la moyenne de toutes les colonnes de \texttt{morpho}. 
On peut récupérer la moyenne de \texttt{aa} uniquement, avec

\begin{knitrout}
\definecolor{shadecolor}{rgb}{1, 1, 1}\color{fgcolor}\begin{kframe}
\begin{flushleft}
\ttfamily\noindent
\hlfunctioncall{aggregate}\hlkeyword{(}\hlsymbol{morpho}\hlkeyword{\usebox{\hlnormalsizeboxdollar}}\hlsymbol{aa}\hlkeyword{,}{\ }\hlargument{by}{\ }\hlargument{=}{\ }\hlfunctioncall{list}\hlkeyword{(}\hlargument{hote}{\ }\hlargument{=}{\ }\hlsymbol{morpho}\hlkeyword{\usebox{\hlnormalsizeboxdollar}}\hlsymbol{sphot}\hlkeyword{,}{\ }\hlargument{parasite}{\ }\hlargument{=}{\ }\hlsymbol{morpho}\hlkeyword{\usebox{\hlnormalsizeboxdollar}}\hlsymbol{sppar}\hlkeyword{)}\hlkeyword{,}\hspace*{\fill}\\
\hlstd{}{\ }{\ }{\ }{\ }\hlsymbol{mean}\hlkeyword{)}\mbox{}
\normalfont
\end{flushleft}
\begin{verbatim}
##    hote parasite      x
## 1  Disa     conf 0.9759
## 2  Sasa     conf 1.2000
## 3  Divu     dipl 1.2000
## 4  Dian     eleg 1.5750
## 5  Disa     eleg 1.5289
## 6  Divu     eleg 1.6381
## 7  Obme     eleg 1.1833
## 8  Dipu     erge 1.9896
## 9  Disa     erge 1.6516
## 10 Divu     erge 1.7800
## 11 Disa     falc 0.9029
## 12 Divu     falc 1.2829
## 13 Dian     frat 1.3333
## 14 Disa     furc 1.7694
## 15 Dipu     igno 1.1818
## 16 Disa     igno     NA
## 17 Divu     igno 1.0137
## 18 Limo     igno 0.9150
## 19 Sasa     igno     NA
## 20 Disa     kech 1.4686
## 21 Divu     kech 1.5722
## 22 Disa     morm 1.4000
## 23 Disa     neif     NA
## 24 Divu     neif 1.3888
## 25 Dipu     ther 1.9996
## 26 Divu     tome 1.8167
\end{verbatim}
\end{kframe}
\end{knitrout}

